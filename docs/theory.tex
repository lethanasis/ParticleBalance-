\documentclass{notes}

\usepackage{amssymb}
\usepackage{hyperref}
\usepackage{siunitx}

\title{Theory of particle and power balance}
\author{Mathias Hoppe}
\date{2023-07-30}

\begin{document}
	\maketitle

	In this document we consider the theory underlining the 0D particle and
	power balance model implemented in the present Python tool. In the model,
	we consider a plasma consisting of six distinct particle species:
	thermal electrons, runaway electrons, main ions, impurity ions, main
	neutrals, and impurity neutrals. In the particle balance, main (impurity)
	ions and neutrals are considered together, while the electron densities are
	deduced from quasi-neutrality.

	\tableofcontents

	\section{Particle balance}
	The ion rate equation, in the notation of the \DREAM\ paper, is
	\begin{equation}\label{eq:ionrate}
		\frac{\dd n_i^{(j)}}{\dd t} =
			\left(
				I_i^{(j-1)}n_e+ 
				\mathcal{I}_i^{(j-1)}
			\right)n_i^{(j-1)} -
			\left(
				I_i^{(j)}n_e +
				\mathcal{I}_i^{(j)} -
				R_i^{(j)}n_e
			\right)n_i^{(j)} +
			R_i^{(j+1)}n_en_i^{(j+1)}.
	\end{equation}
	In the steady-state limit, we take $\dd n_i^{(j)}/\dd t\to 0$.

	\subsection{Newton algorithm}
	It is observed that a fixed-point iteration algorithm can be slow, or even
	fail to converge, in cases when the initial guess for the electron density
	$n_e$ is far from the true solution. To accelerate the convergence, and 
	hopefully improve its robustness, we can formulate the problem in terms of
	Newton iteration towards the solution.

	We introduce the target function
	\begin{equation}\label{eq:target}
		f(\tilde{n}_e) = \tilde{n}_e - \sum_i \sum_{Z_0} Z_0n_i^{(Z_0)},
	\end{equation}
	where $\tilde{n}_e$ is the assumed electron density and $n_i^{(Z_0)}$ are
	the resulting equilibrium ion densities. The ion densities are also
	non-linear functions of $\tilde{n}_e$, which calls for the use of a Newton
	algorithm when searching for the root $f(\tilde{n}_e)=0$.

	Perhaps the most straightforward way of finding $f(\tilde{n}_e)=0$ is to
	construct a single equation system for both the ion and electron density
	equations:
	\begin{equation}
		\mathbb{A}\bb{x} = \bb{b},
	\end{equation}
	where
	\begin{equation}
		\bb{x} =
		\begin{pmatrix}
			n_1^{(0)} & n_1^{(1)} & \ldots & n_e
		\end{pmatrix}^\top
	\end{equation}
	and $\mathbb{A}$ represents the (non-linear) equations for the ion charge
	state and electron density, as well as the total ion particle density
	\begin{equation}\label{eq:Ni}
		N_i = \sum_{ij} n_i^{(j)}.
	\end{equation}
	When evaluating the jacobian matrix $\mathbb{J}$ for $\mathbb{A}$, we must
	therefore evaluate partial derivates of equations~\eqref{eq:ionrate},
	\eqref{eq:Ni} and the second term of~\eqref{eq:target}, with respect to
	$n_i^{(j)}$ and $\tilde{n}_e$. We can neglect the $\tilde{n}_e$ dependence
	of the fast-electron impact ionization $\mathcal{I}_i^{(j)}$, so that a
	partial derivative of eq.~\eqref{eq:ionrate} with respect to $n_i^{(j)}$
	becomes $\mathbb{A}$ itself, while a derivative with respect to
	$\tilde{n}_e$ must be applied to a rate coefficient times $\tilde{n}_e$,
	yielding
	\begin{equation}
		\frac{\partial}{\partial\tilde{n}_e}\left( I_i^{(j-1)}\tilde{n}_e \right)
		=
		\tilde{n}_e
		\frac{\partial I_i^{(j)}}{\partial\tilde{n}_e} +
		I_i^{(j)}.
	\end{equation}
	The row in $\mathbb{J}$ corresponding to $\partial/\partial\tilde{n}_e$ will
	thus appear as
	\begin{equation}
		\tilde{n}_e\left(
			\frac{\partial I_i^{(j-1)}}{\partial\tilde{n}_e} -
			\frac{\partial I_i^{(j)}}{\partial\tilde{n}_e} +
			\frac{\partial R_i^{(j+1)}}{\partial\tilde{n}_e} -
			\frac{\partial R_i^{(j)}}{\partial\tilde{n}_e}
		\right)
		+
		I_i^{(j-1)} - I_i^{(j)} + R_i^{(j+1)} - R_i^{(j)}.
	\end{equation}

	\section{Power balance}
	\subsection{Electron power balance}
	The main power balance we consider is that of the thermal electrons. The
	thermal electron power balance equation is
	\begin{equation}
		\frac{\dd W_e}{\dd t} =
			P_{\rm re} - P_{\rm rad} + \sum_i P_{ei}.
	\end{equation}
	\red{TODO: Also consider ohmic heating power}
	Here, $P_{\rm re}$ denotes the energy gained by thermal electrons via
	collisions with runaway electrons, given by
	\begin{equation}
		P_{\rm re} = 2\pi\int \int_{-1}^1 vF_{\rm fr}(p) f_{\rm re}(p,\xi)\,p^2\,\dd\xi\dd p,
	\end{equation}
	where $v=p/\gamma$ is the electron speed, $f_{\rm re}$ is the runaway
	electron distribution function, and the friction force is, to good
	approximation,
	\begin{equation}
		F_{\rm fr}(p)\approx \frac{e^4 n_e\ln\Lambda_{ee,{\rm rel}}}{4\pi\epsilon_0^2 m_e v^2},
	\end{equation}
	with the relativistic Coulomb logarithm
	\begin{equation}
		\ln\Lambda_{ee,{\rm rel}} =
			14.9 + \ln\left(\frac{T_e}{\SI{1}{keV}}\right) - \frac{1}{2}\ln\left(\frac{n_e}{10^{20}\,\si{\per\meter\cubed}}\right)
			+ \frac{1}{k}\ln\left[ 1 + \left(\frac{\gamma-1}{p_{\rm th}^2}\right)^{k/2}\right]
	\end{equation}
	
	The radiation loss term consists of multiple contributions and can be
	written on the form
	\begin{equation}
		P_{\rm rad} = n_e\sum_i\sum_{j=0}^{Z_i-1}
			n_i^{(j)} L_i^{(j)},
	\end{equation}
	with radiation coefficients
	\begin{equation}
		L_i^{(j)} = L_{\rm line} + L_{\rm free} +
			\Delta W_i^{(j)}\left(I_i^{(j)} - R_i^{(j)}\right),
	\end{equation}
	where $L_{\rm line}$ is the ADAS \emph{PLT} rate, $L_{\rm free}$ is the
	ADAS \emph{PRB} rate, and $\Delta W_i^{(j)}$ is the ionization threshold
	obtained from NIST.

	The inter-species collisional energy transfer $P_{kl}$ is taken from the
	basic result found in e.g.\ Landau-Lifshitz
	\begin{equation}\label{eq:Pkl}
		P_{kl} = 
			\frac{
				\left\langle nZ^2 \right\rangle_k
				\left\langle nZ^2 \right\rangle_l
				e^4\ln\Lambda_{kl}
			}{
				\left(2\pi\right)^{3/2}\epsilon_0^2 m_k m_l
			}
			\frac{T_k - T_l}{
				\left(
					\frac{T_k}{m_k} + \frac{T_l}{m_l}
				\right)^{3/2}
			},
	\end{equation}
	where $\langle nZ^2 \rangle_k = \sum_{j=0}^{Z_k}n_k^{(j)}Z_{0j}^2$ and
	the Coulomb logarithm for collisions between species $k$ and $l$ is given
	by
	\begin{align}
		\ln\Lambda_{ee} &= 14.9 + \ln\left(\frac{T_e}{\SI{1}{keV}}\right)
			- \frac{1}{2}\ln\left(\frac{n_e}{10^{20}\,\si{\per\meter\cubed}}\right),\\
		%
		\ln\Lambda_{ei} &= 17.3 + \frac{3}{2}\ln\left(\frac{T_e}{\SI{1}{keV}}\right)
			- \frac{1}{2}\ln\left(\frac{n_e}{10^{20}\,\si{\per\meter\cubed}}\right).
	\end{align}

	\subsection{Ion power balance}
	For the ion temperatures, we use the power balance equation
	\begin{equation}
		\frac{\dd W_i}{\dd t} =
			\sum_{k\neq i} P_{ik} - P_{ik,{\rm cx}}
	\end{equation}
	where the inter-species collisional energy transfer $P_{ik}$ is as in
	equation~\eqref{eq:Pkl} and runs over both ion and electron species. The
	charge-exchange energy loss term takes the form
	\begin{equation}\label{eq:cx}
		P_{ik,{\rm cx}} =
			\frac{3}{2}n_i^{(0)}\left(T_i - T_i^{(0)}\right)
			R_{ik,{\rm cx}}^{(1)}n_k^{(1)},
	\end{equation}
	where $T_i^{(0)}$ denotes the neutral temperature for species $i$ and
	$R_{ik,{\rm cx}}^{(1)}$ is the charge-exchange rate whereby a neutral ion of
	species $i$ exchanges an electron with a singly charged ion of species $k$.
	Note that ADAS data is only available for charge-exchange involving
	deuterium.

	\subsection{Neutral power balance}
	For neutrals we consider the power balance
	\begin{equation}
		\frac{\dd W_i^{(0)}}{\dd t} =
			\sum_{k\neq i} P_{ik}^{\rm NN} + P_{i,{\rm cx}} - P_{i,\rm diff}.
	\end{equation}
	The first term corresponds to neutral-neutral collisional energy exchange
	and takes the form
	\begin{equation}
		P_{ik}^{\rm NN} =
			n_k^{(0)} r_{ik}^2
			\frac{4\sqrt{\pi}}{3}
			\frac{m_im_k^2}{\left(m_i+m_k\right)^3}
			\sqrt{2\frac{m_kT_i^{(0)} + m_iT_k^{(0)}}{m_im_k}}
			\left(T_k^{(0)}-T_i^{(0)}\right),
	\end{equation}
	with $r_{ik}=(1/2)(r_i+r_k)$ denoting the effective radius for momentum
	scattering, and $r_i$ and $r_k$ are the radii for the two neutral particles.
	The charge-exchange term is~\eqref{eq:cx}, but with a change of sign.

	The last term models cooling of neutrals via heat diffusion to the wall. It
	is
	\begin{equation}
		P_{i,\rm diff} = \frac{D_i}{\Delta r_{\rm w}^2}\left(T_i^{(0)}-T_{\rm wall}\right) n_i^{(0)},
	\end{equation}
	where $\Delta r_{\rm w}$ denotes the characteristic distance over which the
	heat must be transported, and the diffusion coefficient prescribed (it is
	taken to be $D_i=1$-$\SI{10}{m^2/s}$ to best match DIII-D experiments).

	\section{Solving the equation system}\label{sec:solution}
	The equation system described in the previous section comprises a
	multi-dimensional, non-linear system of equations. We are looking for the
	time-asymptotic ($\dd/\dd t\to 0$) solution to these equations, and this
	means that a full non-linear solution of the equations may be challenging,
	due to the potential existence of multiple solutions in this
	multi-dimensional space. The time-asymptotic limit however allows us to make
	a number of simplifications and express the system of equations as a single
	non-linear equation to be solved.

	For a plasma with one main ion species ($D$) and one impurity species ($i$),
	the system of equations is
	\begin{equation}\label{eq:sys}
		\begin{cases}
			P_{\rm re} - P_{\rm rad} + P_{eD} + P_{ei} = 0,\\
			%
			P_{De} + P_{Di} - P_{Di,{\rm cx}} = 0,\\
			%
			P_{ie} + P_{iD} - P_{iD,{\rm cx}} = 0,\\
			%
			P_{Di}^{\rm NN} + P_{Di,{\rm cx}} - P_{D,\rm diff} = 0,\\
			%
			P_{iD}^{\rm NN} + P_{iD,{\rm cx}} - P_{i,\rm diff} = 0.
		\end{cases}
	\end{equation}
	By solving the fourth and fifth equations for $P_{Di,{\rm cx}}$ and
	$P_{iD,{\rm cx}}$, respectively, and substituting the results into the
	second and third equations, and then solving the second and third equations
	for $P_{De}=-P_{eD}$ and $P_{ie}=-P_{ei}$ respectively, the result can be
	substituted into the first equation to yield
	\begin{equation}
		\begin{gathered}
			P_{\rm re} - P_{\rm rad} + P_{Di} + P_{Di}^{\rm NN} - P_{D,\rm diff} +
			P_{iD} + P_{iD}^{\rm NN} - P_{i,\rm diff} = 0,\\
			%
			\implies\\
			%
			P_{\rm re} - P_{\rm rad} - P_{D,\rm diff} - P_{i,\rm diff} = 0.
		\end{gathered}
	\end{equation}
	All terms in this expression depends non-linearly on the electron
	temperature $T_e$, not at least via the ion and electron densities which
	we assume to be in coronal equilibrium. The two transport loss terms however
	also depend on the $D$ and $i$ species neutral temperatures, $T_D^{(0)}$ and
	$T_i^{(0)}$. If we assume that the ion and electron equilibriation is very
	fast, so that $T_e=T_i$, and if we assume that the neutral-neutral
	equilibration time scale is also very fast, so that $T_D^{(0)}=T_i^{(0)}$,
	the fourth and fifth equations of~\eqref{eq:sys} reduce to
	\begin{equation}
		\begin{cases}
			\frac{3}{2}n_D^{(0)} R_{Di,\rm cx}^{(1)}n_i^{(1)}
				\left(T_e-T_D^{(0)}\right) =
			\frac{D_i}{\Delta r_{\rm w}^2}\left(T_D^{(0)}-T_{\rm wall}\right)n_D^{(0)},\\
			%
			\frac{3}{2}n_i^{(0)} R_{iD,\rm cx}^{(1)}n_D^{(1)}
				\left(T_e-T_i^{(0)}\right) =
			\frac{D_i}{\Delta r_{\rm w}^2}\left(T_i^{(0)}-T_{\rm wall}\right)n_i^{(0)},
		\end{cases}
	\end{equation}
	from which closed expressions for the neutral temperatures can be obtained
	as
	\begin{equation}
		\begin{aligned}
			T_D^{(0)} &= \frac{
				\frac{3}{2}n_i^{(1)}R_{Di,\rm cx}^{(1)}T_e +
				\frac{D_i}{\Delta r_{\rm w}^2}T_{\rm wall}
			}{
				\frac{3}{2}n_i^{(1)}R_{Di,\rm cx}^{(1)} +
				\frac{D_i}{\Delta r_{\rm w}^2}
			}
			\equiv (1-\alpha_D) T_e + \alpha_D T_{\rm wall},\\
			%
			T_i^{(0)} &= \frac{
				\frac{3}{2}n_D^{(1)}R_{iD,\rm cx}^{(1)}T_e +
				\frac{D_i}{\Delta r_{\rm w}^2}T_{\rm wall}
			}{
				\frac{3}{2}n_D^{(1)}R_{iD,\rm cx}^{(1)} +
				\frac{D_i}{\Delta r_{\rm w}^2}
			}
			\equiv (1-\alpha_i) T_e + \alpha_i T_{\rm wall},\\
		\end{aligned}
	\end{equation}
	with the transport efficiency $\alpha_k$ given by
	\begin{equation}
		\alpha_k = \frac{
			D_i/\Delta r_{\rm w}^2
		}{
			\frac{3}{2}n_D^{(1)}n_i^{(1)}R_{iD,\rm cx}^{(1)}/n_k^{(1)} +
			D_i/\Delta r_{\rm w}^2
		},
		\qquad k\in\{D, i\},
	\end{equation}
	determining whether the neutral temperature is closer to the electron or
	wall temperature.

	\section{Estimation of time-scales}
	For the simplifications made in section~\ref{sec:solution} to be valid, the
	time scales of all processes must be much faster than the duration of the
	runaway plateau which we are considering. In this section, we will therefore
	estimate the time scales of the various processes modelled.

\end{document}
